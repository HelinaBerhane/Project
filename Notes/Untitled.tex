\newcommand{\HTSwiki}{\cite{HighTempSuperconductor-wiki}}
\newcommand{\RVBwiki}{\cite{ResonatingValenceBondTheory-wiki}}
\newcommand{\SCor}{\cite{StronglyCorrelatedModels-paper}}

\section{Introduction}

\subsection{HighTempSuperconductors}

\subsubsection{Transition Temperatures}
While normal {\quo{superconductors usually have transition temperatures below $30 k$, high temperatures have been observed with transition temperatures as high as $138 k$}}\HTSwiki

\subsubsection{Materials}

\paragraph{Certain cuprates} Compounds of copper and oxygen \HTSwiki \n
{\quo{Strong correlations}} have something to do {\quo{with these compounds}} \SCor

\paragraph{Hydrogen sulfide $H_2 S$} {\quo{under extremely high pressure}} is the {\quo{highest temperature superconductor known to date}}, with $T_c = 203 k$ \HTSwiki

\paragraph{Ceramics}
	\begin{itemize}
		\item {\quo{barium doped lanthanum and copper oxide}} with $T_c \approx 35 k$ \HTSwiki
		\item {\err{list some others?}}
	\end{itemize}

{\qs{I think I’ll be investigating cuprates? But I initially thought it would be ceramics?}}{\err{Find out which ones!}}

\subsubsection{Theories}

\paragraph{Resonating Valence Bond Theory} \RVBwiki

\section{Models}
The {\quo{two simplest models for strongly correlated electrons}} are \SCor
	\begin{itemize}
		\item {\quo{The Hubbard}} model \SCor
		\item {\quo{The $t-J$}} model \SCor
	\end{itemize}

{\quo{P.W Anderson proposed that a resonating valence bond wave-function,}} consisting {\quo{of a superposition of valence bond states, contains the ingredients to account for a consistent theory of}} these models \SCor

{\quo{Variational Monte Carlo}} programs have been successful in {\quo{describing some of the peculiar properties of these cuprates}}

{\na{Cedric’s Paper proposes an extension to these to include {\quo{strongly correlated }} … maybe that’s what he wants me to do?}}

\section{Magnetism}
\subsection{Antiferromagnetism}

\section{DMFT}

\subsection{Aim}
{\quo{There are many interesting unsolved problems, such as high critical temperatures}} \qi

\subsection{Perks}
{\quo{
	The simple Hubbard {\qs{(“truncated” PPP)}} model on a 2D lattice
		\eq{-t \sum_{ij, \sigma}{c^\dagger_{i\sigma}c_{j\sigma}} - \mu \sum_{i,\sigma}{c^\dagger_{i\sigma}c_{i\sigma}} + U \sum_i{n_{i\uparrow} n_{i\downarrow}} }
	gives rise to many electronic phases, e.g.
		\begin{itemize}
			\item anti ferromagnetism
			\item superconductors
		\end{itemize}
}} \qi