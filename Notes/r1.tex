\section{Models}

We need to aproximate things because solving the general Hamiltonian would take an unreasonable amount of data, so we use approximations to make the calculations easier % 26

\subsection{The Hartree Fock Method}

In the Hartree Fock method, the wavefunction {\qs{(of what?)}} is approximated by a Slater determinant, effectively treating the effects of all of the electrons as an averaged potential.

\paragraph{Slater Determinant}
{\na{this is an expression that describes the wave function of a multi-fermionic system that satisfies anti-symmetry requirements and consequently the Pauli principle by changing sign upon exchange of two electrons (or other fermions). \cite{Slater-wiki}}}

However, as each electron would be repelled from each other, there is a motion of electrons relative to each other.
This is called the electronic correlation

Because the hartree fock method ignores this correlation, it consistently gives a large overestimate of the energy.
To combat this, we use {\qs{what exactly?}}

\pagebreak
\section{Electron Correlations}

\pagebreak
\section{Calculation of Electron Correlations}
}}

\pagebreak
\section{Anderson Impurity Model}

In the Anderson Impurity Problem, a single impurity is placed in a bath of {\qs{how many?}} atoms.




\subsection{Monte Carlo Simulations}

\pagebreak

\section{Quantum Monte Carlo Simulation}

{\com{
We're going to use a quantum monte carlo simulation to solve the Anderson Impurity problem \cite{QISolvers}

To do this, we will input a Self-Consistency equation
$$\mathcal{G}=G_i(i\omega_n)+\Sigma(i\omega_n)$$

and the program should output a lattice Green's function
$$G_i(i\omega_n)=\frac{1}{N}\sum_k{\frac{1}{i\omega_n-(\epsilon(k)-\mu)-\Sigma(i\omega_n)}}$$

which can be used to get another self-consistency equation with
$$G_{i\sigma}^{-1}(i\omega_n)+\Sigma_\sigma(i\omega_n)=\mathcal{G}_\sigma(i\omega_n)$$
}}

A quantum monte carlo simulation is one that{\qs{does what?}}

\paragraph{Main steps} In this program, we:
\begin{itemize}
    \item make a one dimensional lattice of $N$ spins, denoted by $\pm 1$
    \item this lattice extends into imaginary time, in l time slices
        {\qs{
        \begin{itemize}
            \item would each spin stretching in imaginary time be the same as the initial spin?
            \item or would each point on each time slice also be randomly generated?
        \end{itemize}
        }}
    \item sweeping across each lattice point, and then across each time slices, the spins will be flipped
    \item whether the spin stays flipped depends on the calculated weight
    \item this weight depends on the spins of all of the points in the lattice, according to this equation
        \begin{equation}
            \\
        \end{equation}
    \item if this weight is above 1, the change in spin is accepted automatically.
    \item if this spin is below 1, a random number is generated.
    \item the flip is then accepted if the weight is above this random number
    \item during this process, different metrics are recorded about the lattice
    \item first, we record the average spins of the lattice points
\end{itemize}

This simulation was written and run to calculate the correlations between electrons.

\bib

\end{document}
