\section{Introduction}



\subsection{Magnetism}

{\quo{Both the static and dynamic magnetic properties of the equilibrium magnetic system have been the most investigated magnetic system}}\SDyn

{\quo{As one of the principle static magnetic properties, the magnetic ordering has been the most widely studied property.}}\SDyn

{\quo{The Curie-Weiss law describes the dependance of magnetisation on temperature in paramagnets.}}\SDyn

This was {\quo{later generalised to describe the phenomenon of ferromagnetism.}}\SDyn

{\quo{magnetic ordering introduces a regular ordered magnetic structure in many crystals,}} which {\quo{means that in the absence of a magnetic field, the mean magneitc moment of at least one atom in each unit cell of the crystal $\ne 0$}} \SDyn

\paragraph{Ferromagnets}
{\quo{In ferromagnets, the mean magnetic moments of all the atoms have the same orientation, provided that the temperature of the ferromagnetic does not exceed a critical {\na{Curie}} temperature $T_c$}}\SDyn
{\quo{for this reason, ferromagnets have spontaneous magnetic moments (i.e. $\ne 0$ macroscopic magnetic moments) even in the absence of an external field}}\SDyn

\paragraph{Anti-ferromagnets}
{\quo{The mean atomic magnetic moments compensate each other within each unit cell in zero external magnetic field.}}\SDyn
i.e. {\quo{an antiferromagnet consists of a set of antiparallel magnetic sub lattices, each of which has a non-zero mean magnetic moment.}}\SDyn

{\quo{this occurs if the temperature of the antiferromanget is lower than a critical {\na{Neil}} temperature $T_N$}}\SDyn

\paragraph{Ferrimagnetism}
{\quo{There exists a number of antiparallel magnetic sub lattices whose magnetic moments are uncompensated in contrast to anti-ferromagnetism}}\SDyn
$\therefore$ {\quo{ferrets exhibit spontaneous net magnetic moments, even}} without an {\quo{external magnetic field.}}
{\err{continue from \SDyn page 2.}}

\subsection{High Temperature Superconductors}

\subsubsection{Cuprates}
{\quo{Despite the apparent complexity of different cuprates, they all have two dimensional $CuO_2$ planes.}}\SCor

{\quo{In essence, all high temperature superconductors consist of these two dimensional planes, sandwiched between intervening atomic layers.}}\SCor

{\quo{These layers are composed mostly out of alkaline-earths, rare earths, oxygen and halogenides.}}\SCor

{\quo{Depending on the number of $CuO_2$ planes per unit cell, the materials have a single, double or triple-plane form.}}\SCor

{\quo{It’s widely accepted that the $CuO_2$ planes host electronic excitations which are most relevant to superconductivity.}}\SCor

{\quo{However, there’s still some controversy about the role of the co-called apical oxygen atoms, located above and below the $CuO_2$ plane}}\SCor
{\qs{\begin{itemize}
	\item define “co-called”
	\item define “apical”
\end{itemize}}}

{\quo{It’s experimentally well established that the parent compounds are charge transfer insulators and}} are {\quo{ordered anti-ferromagnetically below a {\na{Neil}} temperature $T_n$.}}

{\quo{Below the temperature $T_n$, the unpaired holes of the $Cu^{2+}$ ions are anti-ferromagnetically coupled via super-exchange through the $O^{2-}$}}
{\qs{\begin{itemize}
	\item define “anti-ferromagnetism”
	\item define “magnetically coupled”
	\item define “super-exchange”
	\item define $Cu^{2+}$ or $O^{2-}$
\end{itemize}}}

{\quo{The picture of a magnetic insulator contradicts the simple band-structure point of view.}}
{\qs{\begin{itemize}
	\item define “the simple band-structure point of view”
\end{itemize}}}

\subsection{Models}

\subsubsection{Ising Model}

{\quo{The Ising model consists of spins }} $(\pm 1)$ {\quo{which are confined to the sites of a lattice.}}\guide

{\quo{These spins interact with their nearest neighbours on the lattice with interaction constant $\mathcal{J}$.}}\guide

{\quo{The Hamiltonian for this model}} is given by:
\begin{align*}
	\mathcal{H} =&\; -\mathcal{J}\sum_{i, j}{\sigma_i \sigma_j} - H\sum_i{\sigma_i}\\
	\sigma_i =&\; \pm 1\\
\end{align*}

This {\quo{model has been solved exactly in one dimension,}} so it’s {\quo{known that there is no transition.}}

{\quo{In two dimensions, }} it has been shown {\quo{that there is a second order phase transition with divergences in the :}} \guide
\begin{itemize}
	\item specific heat
	\item susceptibility
	\item correlation length
\end{itemize}

{\quo{The Ising model can be simulated using simple sampling techniques: spin configurations can be generated randomly with their contribution weighted by a Boltzmann factor.}} \guide
{\quo{Unfortunately, most of the configurations which are produced in this}} way {\quo{will contribute relatively little to the equilibrium averages.}}\guide
{\quo{\na{more sophisticated methods are required if we’re to obtain results of sufficient accuracy to be useful.}}}

|———————————————————————————————|


\section{Calculations}

\subsection{Weight}
Whether

\subsection{Partition Function}
The partition function is defined as

\subsection{Greens Function}
In order to calculate the greens function,

\subsection{Hirsch Fye Quantum Monte Carlo Algorithm}

In order to solve the Anderson impurity model
A quantum Monte Carlo algorithm was written in order to

\subsubsection{Algorithm}

\paragraph{General Structure}

The algorithm generated a 2 dimensional lattice of spins, with
\begin{itemize}
	\item the width $=$ the number of physical lattice points
	\item the length $=$ the number of time slices

Then, it would pass through the lattice, going to each lattice point in a time slice in order, before moving on to the next time slice.

For each point in the lattice, it would flip the spin and calculate the probability of the spin staying flipped.

This probability was calculated as the ratio of the weight after the flip over the weight before the flip.

If:
\begin{itemize}
	\item $p>1$, the flip would be automatically accepted
	\item $0<p<1$, a random number would be used as the upper bound for acceptance
	\item $p<0$,
\end{itemize}

Once it has swept through the entire lattice, it would start again at the beginning, and would continue for {\err{x}} iterations.

\paragraph{Initial Conditions}

The algorithm needed 3 input conditions, from which the rest could be calculated.
These were:
\begin{itemize}
	\item $U = $ the
	\item $\beta = \frac{1}{T}$
		\item where $T = $ the temperature
	\item
\end{itemize}



\paragraph{Finite Size Effects}

{\quo{The effects of the finiteness of the system could be dramatic.}}\guide

{\quo{Since our interest is in determining the properties of the corresponding infinite system,}} we need {\quo{some sound, theoretically based methods for extracting such behaviour for the results obtained on the finite system.}} \guide

{\quo{One fundamental difficult which arises in interpreting simulations data is that the equilibrium, thermodynamic behaviour of a finite system is smooth as it passes through a phase transition.}} \guide
{\err{Do we even need to think about phase transitions? If not, return to page 77 of \guide}}

\paragraph{unknown tidbits}
\begin{\itemize}
	\item {\quo{the correlation length is limited by the lattice size $L$}}\guide
\end{itemize}
