%---- initialisation ----%
\documentclass{article}

% packages
\usepackage[utf8]{inputenc}
\usepackage{amssymb}
\usepackage[]{amsmath, amsfonts, graphicx}
\usepackage{titlesec}
\usepackage{graphicx}
\usepackage{fancyhdr}
\usepackage{hyperref}
\usepackage[usenames, dvipsnames]{color}
\usepackage{parskip}

% commands
\definecolor{LightGrey}{gray}{0.7}

\newcommand{\summary}{1}

\newcommand{\com}[\summary]{\color{CadetBlue}}
\newcommand{\err}[\summary]{\color{Red}\textbf}
\newcommand{\na}[\summary]{\color{LightGrey}}
\newcommand{\qs}[\summary]{\color{Plum}}
\newcommand{\imp}[1]{{\color{Mahogany}\textbf{#1}}}
\newcommand{\ask}[\summary]{\color{Plum}\textbf}
\newcommand{\code}{\color{CadetBlue}}
\newcounter{an}
\setcounter{an}{0}
\newcommand{\anki}[1]{\addtocounter{an}{1} \textbf{Anki Question \arabic{an} : #1}}

\newcounter{qu}
\setcounter{qu}{0}
\newcommand{\que}[1]{\addtocounter{qu}{1} \textbf{Question \arabic{qu} : #1}}

% margins
\addtolength{\oddsidemargin}{-.875in}
\addtolength{\evensidemargin}{-.875in}
\addtolength{\textwidth}{1.75in}
\addtolength{\topmargin}{-0.75in}
\addtolength{\textheight}{1.875in}
\setlength{\parskip}{1em}

% fancyhdr
\pagestyle{fancy}
\fancyhf{}
\rhead{Helina Berhane}
\lhead{Complex Matrix Manipulation Algorithms}
\rfoot{Page \thepage}

%---- information ----%
\title{Complex Matrix Manipulation Algorithms}
\date{\vspace{-8ex}}

%---- main text ----%
\begin{document}
\maketitle
\tableofcontents
\pagebreak

\section{Exponentials}
$$e^a=\sum_{n=1}^{\infty}{\frac{x^n}{n!}}$$

One way to calculate this would be to calculate the factorials and powers separately before adding them together.
This could be done with a for loop with:
{\code{}}
or recursively with:
{\code{}}

One problem with this is that there is a limit to the size of a number that a computer can store (with simple methods), and even using the {\code{long long int}} type, the largest factorial that can be calculated is fact(15), meaning you can only do 15 iterations, giving a result with an error of {\err{?}} \%.

Alternatively, you could do each division of the factorial immediately after the power, reducing the overall size of the numbers used in each step:
\begin{equation}\begin{split}
    e^A = &\; \sum_{n=1}^\infty{step^n}     \\
    step^n = &\; step^{n-1}\times\frac{A}{n}\\
    step^1 = &\; 1                          \\
\end{split}\end{equation}
\end{document}
